\documentclass{article}
\usepackage{ben}

\begin{document}
\newcommand{\phd}{Ph.D.}
\newcommand{\university}{%
\texttt{<your-college-here>}
%%  Carnegie Mellon University
%%  Chalmers
%%  Cornell University
%%  EPFL
%%  Harvard
%%  MIT
%%  Northeastern University
%%  The University of Wisconsin
%%  UPenn
%%  UT Austin
%%  Urbana Champagne
}

%% TODO tips from Jonathan
%% -- sound attractive. Tell how you're in demand
%% -- give numbers. All about numbers. For research, for TA

\newcommand{\sloc}{7.7 million lines}
\newcommand{\numprojects}{43}

%% Research statement, not personal statement

%% 1. Describe areas of research that interest me. Why?
%%    - Used to filter application, to choose which Professor should read it
%% 2. Describe research projects. What was the goal? Why was it important? 
%%    What tried? What worked? What learned?
%% 3. ditto
%% 4. ditto
%% 5. Why do I need a PhD?
%% 6. Why CMU?
%%    - What Professors appeal to you, what papers were fun to read? 
%%    Why's it the right place?

%% I could start with my story, with how I finally know what I want to be when I grow up
%% -- but what do I want to be? A professor is that it? Well they want researchers
%% Else I have this weirdo intro about what research I want to do. Which could just be less weird

%% HEY am I more enthused with the idea of research than with research? 
%% 2013-10-28: Yes, but hopefully not to the extent of bpl. I've usually been good about actually going for things (moving out) instead of just complaining (about parents).

\section{Introduction}
%% 2013-10-16: Statement is my own words. Not Adam's, not Amal's, not Fabian's not Ross's not Bill's. Mine.
%% GOAL: say you want to do PL research. Give a sort of research statement. 
%% Initial sentences should lead in to me wanting to study PL
Computer programming ought to be a delightful, empowering activity. 
Not an occupation, not a chore, task, or routine, but an activity.
Too often, it is not. 
%% inability to express certain ideas, lack of fluency, usefullness, 
Language barriers like cryptic error messages, insufficient freedom of expression, and poor feedback mechanisms prevent users from accomplishing their goals in an enjoyable or timely manner.
This is a problem.
The machines we have at our disposal have incredible raw computing strength, but unless humans can interface with and direct this potential it will remain dormant. 
%% Second, it's just plain inspiring to work with a computer and find yourself doing great things just like that.
I wish to study programming languages so that I may develop tools that will keep computer programming fun, refreshing, and productive.
%% %% art vs. science is too general. Way too general
%% To develop it further as both an art\textemdash a means of creative expression, a conduit for realizing ideas and running experiments\textemdash and a science\textemdash a formalized procedure for solving difficult problems or organizing large projects.

\section{Current Research}
%% Slide into current research, eliding need for me to say exactly what I wanna do.
These tools can take a wide variety of forms. 
Incremental compilation, version control, and unit testing are indispensible tools to modern developers.
On the horizon are proof assistants, which offer rigorous and increasingly easy-to-build formal proofs of correctness.
%% In the near future, proofs will take the place of unit tests in large-scale systems.
Additionally, new features are constantly being suggested and added to improve existing languages.

Here, my current research provides an example.
%% Describe project, high level. TODO ask Ross what level of detail is OK
Since Summer 2013, I have worked with Professor Ross Tate on formalizing a new paradigm for object-oriented languages.
The proposal is to define a new, broad division among objects, orthogonal to the separation between classes and interfaces.
Just as an interface or abstract class provides the signature for fully implemented classes, Professor Tate observed that classes and interfaces like Java's \texttt{Comparable} describe how objects are manipulated with other objects.
One never seeks to use, for example, a list of object \texttt{Comparable}.
However, one often wants a list of some generic type implementing that interface.
Hence there is a distinction between classes and interfaces in the traditional sense and classes and interfaces which are used to bound type parameters; furthermore, enforcing this distinction yields useful results for the entire type system.

My involvement began as a summer opportunity to experience research and has since translated into a major role as part of a PLDI submission.
Over the summer, I compiled and analyzed existing Java code to determine whether the division Professor Tate had observed in isolated cases was in fact a widespread pattern.
First, I wrote a small program to graph and analyze the inheritance hierarchy of a Java project, identifying any cycles created by subtyping definitions.
%% The message, before I get to the footnote, is that I wrote this shit myself. I had help _ inspiration but I wrote it
This program consisted of an abbreviated execution of the OpenJDK compiler, to obtain the inheritance hierarchy, and a Python script to create the inheritance graph and find the cycles.\footnote{In retrospect, the NetworkX library would have been completely sufficient. Still, finding and implementing the cycle-finding algorithm was a pleasure.}
Then I modified the open JDK compiler to log a notification whenever the source code broke invariants devised by Professor Tate.
\numprojects\ projects and \sloc\ of code later, we found only a few instances where the code broke our assumptions.
However these ``violations'', instead of presenting cause for alarm, affirmed our conviction in the proposed features\textemdash in every occasion, flagged code was attempting to implement our proposed feature.
%% Programmers had already identified and attempted to correct the insufficiency, albiet in ad-hoc locations and use cases.

Presently, Professor Tate, \phd\ student Fabian M\"{u}lb\"{o}ck, and I have been working to prove theorems about type systems under the new paradigm and writing a paper summarizing these result thus far.
Fabian and I developed, with considerable help from Professor Tate, a suitable measure on subtyping judgements.
At this point we are working more independently on writing the first revisions of the paper.
I do not assert to have played a significant role in the identifying the problem, formulating the solution, or proving the results.
Professor Tate had already been working on this project when he gave me the opportunity to work with him, and has offered invaluable feedback correcting my understandings and keeping me motivated and productive while performing the static analysis and developing the measure.
However, I feel that I have contributed greatly in turning ideas into readable, communicative English.
Once I have fully grasped a concept, I take pride in deing able to explain it well to another person.
This, I feel, is one of my greater strengths, and one which I seek to further develop.

\section{Teaching}
%% transition into 3110
Indeed, one of the most valuable experiences from my undergraduate career has been working as staff for cs311, a first course in functional programming.
%% What did I do
During the 4 semesters I have been on staff, I have led recitations for a section of approximately 30 regularly attending students, played a large role in the release of 5 problem sets and the testing of 11, suggested questions for 8 exams, managed the staff version control repository, completely rewriten the automated test harness, and spent many hours grading.
%% What did I learn
%% resposiblity 
%% time management
%% endurance
%% patience
%% how to explain things
%% better listener

Above all else, working on staff has taught me the difference between ideas and concrete actions.
Case in point, the bane of our automated grading scripts have always been compilation errors in student code.
They require manual debugging and disrupt our grading workflow. 
Last fall, a staff member suggested we ask students to submit a patch to their code which we could apply quickly, and then grade as normal.
A fine idea, ready for adoption but for a few scheduling issues; however, it was not until a year later, when I rewrote the test harness, that our pipeline incorporated the generation and delivery of emails to students and a mechanism for verifying patches.
The entire staff had agreed it was a good policy, but we preferred, myself included, to suffer another year of compile errors unabated.
Someone had to build the feature.
In retrospect I am glad I got to it first\textemdash I have learned a lot from that project\textemdash still it was a jarring reminder of the distance between a theory and its practical application.

\section{Why a \phd}
%% What do they wanna hear? That I want a phd to do research?
%% I  think the sentence on teaching should go, but it make a nice hook. 2013-10-29 you live for now.
Teaching has been a wonderful and rewarding experience, and I hope to one day teach at the university level.
That provides one motivation for seeking \phd.
But while teaching inspires me, it is not what drives me.

Before working this summer with Ross, I had only a faint, admittedly romanticized idea of what research was like.
My initial goal with the project was just to try something different, while the opportunity was available, and get an industry job the following year.
As it turned out, I enjoy research. 
I like the independent nature of research, and I value the collaborative demeanour of the computer science community.
Furthermore I enjoy the diverse quality of research. 
The mix between theorizing, experimenting, writing, discussing, rewriting, and evaluating is energizing.
I have found that I thrive in a high-energy environment where myraid skills and imagination come together, where new ideas are encouraged and learning is paramount.
Classes and textbooks now appear to me as living resources that offer guidance for answering unsolved questions rather the sources of pointless exercises assigned for the purpose of building character.
The problems we have worked on and the problems my colleagues here are working on are exciting and motivating.
We are not building popular or profitable things, but correct and beautiful things.
I want to continue in this line of work, to answer the demands for more streamlined programming outlined in my introduction.
These things in mind, I would be doing myself a disservice if I did strive to remain in an academic environment, contributing with a \phd.

I want \texttt{<this-phd-program>} in particular because \_\_\_\_\_\_\_.
%% all the great things about \university
%% TODO moar
%% Hence I would be doing myself a tremendous disservice if I did not pursue a \phd.

%% %% Ending
%% %% TODO comes off as pompous, wanna channel more friendly
%% %% hehehe
%% If I have misjudged the situation, please correct me. 
%% If \university is in fact a strictly bureaucratical institution where new and challenging ideas are sometimes ignored in favor of popular, safe, or profitable alternatives, do let me know.
%% If research and teaching are secondary motives, please tell me, so I can make an informed decision.
%% Otherwise, 
Hope to join you in the Fall.

\end{document}
